
\documentclass[10pt,journal,letterpaper,compsoc, draftclsnofoot, onecolumn]{IEEEtran} %draft mode for double-spaced
\usepackage{graphicx}
\usepackage{subfig} % for horizontal figures
\usepackage{amsmath}
\ifCLASSOPTIONcompsoc
  % IEEE Computer Society needs nocompress option
  % requires cite.sty v4.0 or later (November 2003)
  % \usepackage[nocompress]{cite}
\else
  % normal IEEE
  % \usepackage{cite}
\fi
\usepackage{cite}


\begin{document}

% paper title
% can use linebreaks \\ within to get better formatting as desired
\title{Title Here}


\author{Forrest~N.~Iandola,~\IEEEmembership{Student Member,~IEEE,}
        Praveen~Jayachandran,~\IEEEmembership{Member,~IEEE,}
        and~Tarek~Abdelzaher,~\IEEEmembership{Member,~IEEE}% <-this % stops a space
\IEEEcompsocitemizethanks{
\IEEEcompsocthanksitem Forrest N. Iandola is with the Department of Computer Science, University of Illinois at Urbana-Champaign, Urbana, IL, 61801.
\protect\\
E-mail: iandola1@illinois.edu
\IEEEcompsocthanksitem Praveen Jayachandran is with IBM Research, India.
\protect\\
E-mail: prjayach@in.ibm.com
\IEEEcompsocthanksitem Tarek Abdelzaher is with the Department of Computer Science, University of Illinois at Urbana-Champaign, Urbana, IL, 61801.
\protect\\
E-mail: zaher@illinois.edu}% <-this % stops a space
\thanks{}}

% The paper headers
\markboth{IEEE Transactions on Computers}%
{Iandola, Jayahandran, and Abdelzaher: Optimizing the Structural Robustness of Real-Time Distributed Systems Towards Uncertainties in Execution Time}
% The only time the second header will appear is for the odd numbered pages
% after the title page when using the twoside option.

% for Computer Society papers, we must declare the abstract and index terms
% PRIOR to the title within the \IEEEcompsoctitleabstractindextext IEEEtran
% command as these need to go into the title area created by \maketitle.
\IEEEcompsoctitleabstractindextext{%
\begin{abstract}
Hello I'm an abstract.

\end{abstract}

% Note that keywords are not normally used for peer review papers.
\begin{keywords}
\noindent
C.3 [Special-purpose and Application-based Systems]: Real-time and embedded systems \\
C.2.4 [Computer-communication Networks]: Distributed Systems \\
C.2.2 [Computer-communication Networks]: Network Protocols -- Routing protocols \\
C.2.8 [Mobile Computing]: Algorithm/protocol design and analysis \\

\noindent
Robustness, Real-Time Systems, Distributed Systems, Networks, Delay Uncertainties, Execution Overruns.
\end{keywords}}

\maketitle
\IEEEdisplaynotcompsoctitleabstractindextext
\IEEEpeerreviewmaketitle

\section{Introduction}

\begin{figure}[htb]
  \centering
  \fbox{
      \subfloat[]{\label{fig:delayAmtRandomized_proportionalToPriority}  TODO }    
  }
  \fbox{            
      \subfloat[]{\label{fig:delayAmtRandomized_random} TODO}
  }
  \caption{Influence of extent to which tasks are delayed on robustness, using randomized WCET overrun percentages. Left: importance equal to priority; Right: random importance values.}
  \label{fig:delayAmtRandomized_doubleFigure}
\end{figure}

I'm trying to reference the ``main" number (not $a$ or $b$) of a subfloat figure.
Figure~\ref{fig:delayAmtRandomized_doubleFigure}.
AHA! The trick was to put label{} after caption{} in the figure.

\vspace{-.5in}
\begin{IEEEbiographynophoto}{Forrest Iandola}
earned the BS degree in Computer Science at University of Illinois at Urbana-Champaign in 2012, and he is now pursuing his PhD in Computer Science at University of California, Berkeley. 
His BS thesis focused on scheduling and routing methodologies for real-time distributed systems. 
His PhD research, funded by the National Defense Science and Engineering Graduate (NDSEG) Fellowship, focuses on designing fast and scalable computer vision algorithms for real-time sensing and interactive systems. 
Forrest is also interested in scientific computing, and he designed large-scale physics simulations at three US Department of Energy National Laboratories.
He is the recipient of the Daniel L. Slotnick Scholarship from the UIUC Department of Computer Science, and he is a member of IEEE, SIAM, and ACM.
\end{IEEEbiographynophoto}
\vspace{-.7in}
\begin{IEEEbiographynophoto}{Praveen Jayachandran}
is a research staff member of the next generation
services and cloud computing team at IBM Research, India. His research
interests are in the area of analysis, performance management, and
optimization of distributed systems and networks. He received his PhD from
the University of Illinois at Urbana-Champaign in 2010. He is the recipient
of the best paper award at ECRTS 2009, the best student paper award at
ECRTS 2007, and the C.L. and Jane Liu award from the Department of Computer
Science at UIUC in 2007.
\end{IEEEbiographynophoto}
\vspace{-.7in}

\begin{IEEEbiographynophoto}{Tarek Abdelzaher}
%Tarek's bio from "Efficient flow-control algorithm cooperating with energy allocation scheme for solar-powered WSNs" paper
received his B.Sc. and M.Sc. degrees in Electrical and Computer Engineering from Ain Shams University, Cairo, Egypt, in 1990 and 1994 respectively. 
He received his Ph.D. from the University of Michigan in 1999 on Quality of Service Adaptation in Real-Time Systems. 
He has been an Assistant Professor at the University of Virginia, where he founded the Software Predictability Group until 2005. 
He is currently a Full Professor at the Department of Computer Science, the University of Illinois at Urbana Champaign. 
He has authored/coauthored more than 150 refereed publications in real-time computing, distributed systems, sensor networks, and control. 
He is Editor-in-Chief of the Journal of Real-Time Systems, an Associate Editor of the IEEE Transactions on Mobile Computing, IEEE Transactions on Parallel and Distributed Systems, the ACM Transaction on Sensor Networks, and the Ad Hoc Networks Journal. 
%He was Program Chair of RTAS 2004, RTSS 2006, IPSN 2010 and ICDCS 2010, as well as General Chair of RTAS 2005, IPSN 2007, RTSS 2007, DCoSS 2008, and Sensys 2008. 
Abdelzaher's research interests lie broadly in understanding and controlling performance and temporal properties of networked embedded and software systems in the face of increasing complexity, distribution, and degree of embedding in an external physical environment. 
Tarek Abdelzaher is a member of IEEE and ACM.
\end{IEEEbiographynophoto}

\end{document}


